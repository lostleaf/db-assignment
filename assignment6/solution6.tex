\documentclass[a4paper,11pt]{article}
\usepackage{amsmath,amsthm,amsfonts,amssymb,bm} 
\usepackage{graphicx,psfrag} 
\usepackage{fancyhdr}
\usepackage{color} 
\usepackage{geometry}
\usepackage{multirow}
\usepackage{listings}
\usepackage{enumerate}
\usepackage{leftidx} 
\usepackage{mathrsfs} 
\usepackage{xeCJK}
 
\usepackage{listings}
\usepackage{color}
 
\definecolor{mygreen}{rgb}{0,0.6,0}
\definecolor{mygray}{rgb}{0.5,0.5,0.5}
\definecolor{mymauve}{rgb}{0.58,0,0.82}
 
\lstset{ %
  backgroundcolor=\color{white},   % choose the background color; you must add \usepackage{color} or \usepackage{xcolor}
  basicstyle=\footnotesize,        % the size of the fonts that are used for the code
  breakatwhitespace=false,         % sets if automatic breaks should only happen at whitespace
  breaklines=true,                 % sets automatic line breaking
  captionpos=b,                    % sets the caption-position to bottom
  commentstyle=\color{mygreen},    % comment style
  deletekeywords={...},            % if you want to delete keywords from the given language
  escapeinside={\%*}{*)},          % if you want to add LaTeX within your code
  extendedchars=true,              % lets you use non-ASCII characters; for 8-bits encodings only, does not work with UTF-8
  keepspaces=true,                 % keeps spaces in text, useful for keeping indentation of code (possibly needs columns=flexible)
  keywordstyle=\bfseries,       % keyword style
  language=SQL,                 % the language of the code
  morekeywords={*,...},            % if you want to add more keywords to the set
  numbers=none,                    % where to put the line-numbers; possible values are (none, left, right)
  numbersep=5pt,                   % how far the line-numbers are from the code
  numberstyle=\tiny\color{mygray}, % the style that is used for the line-numbers
  rulecolor=\color{black},         % if not set, the frame-color may be changed on line-breaks within not-black text (e.g. comments (green here))
  showspaces=false,                % show spaces everywhere adding particular underscores; it overrides 'showstringspaces'
  showstringspaces=false,          % underline spaces within strings only
  showtabs=false,                  % show tabs within strings adding particular underscores
  stepnumber=2,                    % the step between two line-numbers. If it's 1, each line will be numbered
  stringstyle=\color{mymauve},     % string literal style
  tabsize=2,                       % sets default tabsize to 2 spaces
  title=\lstname                   % show the filename of files included with \lstinputlisting; also try caption instead of title
}
 
\geometry{left=3.17cm,right=3.17cm,top=2.54cm,bottom=2.54cm}
 
\begin{document}
 
\pagestyle{fancy}
\rfoot{\thepage}
\rhead{\bfseries Database System Concept}
\setlength{\parskip}{0.7ex plus0.2ex minus0.2ex}
\cfoot{\empty}
\lhead{\empty}
 
 
\title{Assignment 6}
\author{Qinglin Li, 5110309074}
\date{}
\maketitle
 
\headheight 3pt
\thispagestyle{fancy}
\section*{Problem 1}
RAID level 1.Because rebuilding in this case just involves copying data from the failed disk's mirror. In the other levels, rebuilding involves reading the extra contents of other disks.

\section*{Problem 2}
By allocating related records to blocks, we can often retrieve most of the requested records by a query with one disk access. So this strategy reduces the number of disk accesses for a given operation, and significantly improves performance.

\section*{Problem 3}
\begin{enumerate}
\item 
Advantages of storing a relation as a file include using the file system provided by the OS, thus simplifying the DBMS, but incurs the disadvantage of restricting the ability of the DBMS to increase performance by using more sophisticated storage structures.
\item
By using one file for the entire database, these complex structures can be implemented through the DBMS, but this increases the size and complexity of the DBMS. 
\end{enumerate}

\section*{Problem 4}
It is preferable to use a dense index instead of a sparse index when the file is not sorted on the indexed field (such as when the index is a secondary index) or when the index file is small compared to the size of memory

\section*{Problem 5}
Hashing distributes search key values uniformly and randomly across the set of buckets available. Because key values do not occupy consecutive buckets, searching for all values within a range may well require reading every bucket --- far more inefficient than a B+-tree or even a sorted file without index.

\section*{Problem 6}
The existence bitmap for a relation can be calculated by taking the union 
(logical-or) of all the bitmaps on that attribute, including the bitmap for 
value null. 
\end{document}
