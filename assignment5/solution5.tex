\documentclass[a4paper,11pt]{article}
\usepackage{amsmath,amsthm,amsfonts,amssymb,bm} 
\usepackage{graphicx,psfrag} 
\usepackage{fancyhdr}
\usepackage{color} 
\usepackage{geometry}
\usepackage{multirow}
\usepackage{listings}
\usepackage{enumerate}
\usepackage{leftidx} 
\usepackage{mathrsfs} 
\usepackage{xeCJK}
 
\usepackage{listings}
\usepackage{color}
 
\definecolor{mygreen}{rgb}{0,0.6,0}
\definecolor{mygray}{rgb}{0.5,0.5,0.5}
\definecolor{mymauve}{rgb}{0.58,0,0.82}
 
\lstset{ %
  backgroundcolor=\color{white},   % choose the background color; you must add \usepackage{color} or \usepackage{xcolor}
  basicstyle=\footnotesize,        % the size of the fonts that are used for the code
  breakatwhitespace=false,         % sets if automatic breaks should only happen at whitespace
  breaklines=true,                 % sets automatic line breaking
  captionpos=b,                    % sets the caption-position to bottom
  commentstyle=\color{mygreen},    % comment style
  deletekeywords={...},            % if you want to delete keywords from the given language
  escapeinside={\%*}{*)},          % if you want to add LaTeX within your code
  extendedchars=true,              % lets you use non-ASCII characters; for 8-bits encodings only, does not work with UTF-8
  keepspaces=true,                 % keeps spaces in text, useful for keeping indentation of code (possibly needs columns=flexible)
  keywordstyle=\bfseries,       % keyword style
  language=SQL,                 % the language of the code
  morekeywords={*,...},            % if you want to add more keywords to the set
  numbers=none,                    % where to put the line-numbers; possible values are (none, left, right)
  numbersep=5pt,                   % how far the line-numbers are from the code
  numberstyle=\tiny\color{mygray}, % the style that is used for the line-numbers
  rulecolor=\color{black},         % if not set, the frame-color may be changed on line-breaks within not-black text (e.g. comments (green here))
  showspaces=false,                % show spaces everywhere adding particular underscores; it overrides 'showstringspaces'
  showstringspaces=false,          % underline spaces within strings only
  showtabs=false,                  % show tabs within strings adding particular underscores
  stepnumber=2,                    % the step between two line-numbers. If it's 1, each line will be numbered
  stringstyle=\color{mymauve},     % string literal style
  tabsize=2,                       % sets default tabsize to 2 spaces
  title=\lstname                   % show the filename of files included with \lstinputlisting; also try caption instead of title
}
 
\geometry{left=3.17cm,right=3.17cm,top=2.54cm,bottom=2.54cm}
 
\begin{document}
 
\pagestyle{fancy}
\rfoot{\thepage}
\rhead{\bfseries Database System Concept}
\setlength{\parskip}{0.7ex plus0.2ex minus0.2ex}
\cfoot{\empty}
\lhead{\empty}
 
 
\title{Assignment 5}
\author{Qinglin Li, 5110309074}
\date{}
\maketitle
 
\headheight 3pt
\thispagestyle{fancy}

\section*{Problem 1}
\begin{itemize}
\item
\textbf{Repetition of Information} is a condition in a relational database where the values of one attribute are determined by the values of another attribute in the same relation, and both values are repeated throughout the relation.

\item
\textbf{Inability to represent information} is a condition where a relationship exists amoung only a proper subset of the attributes in a relation.
\end{itemize}

\section*{Problem 2}
The three design goals are lossless-join decompositions, dependency preserving decompositions, and minimization of repetition of information. They are desirable so we can maintain an accurate database, check correctness of updates quickly, and use the smallest amount of space possible.

\section*{Problem 3}
4NF is more desirable than BCNF because it reduces the repetition of information.

\section*{Problem 4}
The goal is to prove $\alpha\rightarrow \beta\gamma \Longrightarrow \alpha\rightarrow\beta \wedge\alpha\rightarrow\gamma$\par
By axiom of reflexivity, we obtain $\beta\gamma\rightarrow\beta$ and $\beta\gamma\rightarrow\gamma$\par
By axiom of transitivity and $\alpha\rightarrow \beta\gamma$, we obtain $\alpha\rightarrow\beta$ and $\alpha\rightarrow\gamma$

\section*{Problem 5}
Attribute closure:\par
$A \rightarrow ABCDE$\par
$B \rightarrow BD$\par
$C \rightarrow C$\par
$D \rightarrow D$\par
$E \rightarrow ABCDE$\par
$AB \rightarrow ABCDE$\par
$AC \rightarrow ABCDE$\par
$AD \rightarrow ABCDE$\par
$AE \rightarrow ABCDE$\par
$BC \rightarrow ABCDE$\par
$BD \rightarrow BD$\par
$BE \rightarrow ABCDE$\par
$CD \rightarrow ABCDE$\par
$CE \rightarrow ABCDE$\par
$DE \rightarrow ABCDE$\par
$ABC \rightarrow ABCDE$\par
$ABD \rightarrow ABCDE$\par
$ABE \rightarrow ABCDE$\par
$ACD \rightarrow ABCDE$\par
$ACE \rightarrow ABCDE$\par
$ADE \rightarrow ABCDE$\par
$BCD \rightarrow ABCDE$\par
$BDE \rightarrow ABCDE$\par
$CDE \rightarrow ABCDE$\par
$ABCD \rightarrow ABCDE$\par
$ABCE \rightarrow ABCDE$\par
$ABDE \rightarrow ABCDE$\par
$ACDE \rightarrow ABCDE$\par
$BCDE \rightarrow ABCDE$\par
It's easy to find the candidate keys are $A$, $BC$, $CD$, $E$.

\section*{Problem 6}
Assume we have a relation $R(person, hobby, research\_interest)$.\par
\begin{tabular}{c|c|c}
\hline 
person & hobby & research\_interest \\ 
\hline 
Alice & badminton & AI \\ 
Alice & tennis & database\\
Alice & badminton & database\\
Alice & tennis & AI\\
Bob & badminton & AI\\
\hline 
\end{tabular} 

There is no functional dependency so R is in BCNF. But there is a multivalued dependency $person \twoheadrightarrow hobby$ so R is not in 4NF.

\end{document}