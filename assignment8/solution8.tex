\documentclass[a4paper,11pt]{article}
\usepackage{amsmath,amsthm,amsfonts,amssymb,bm} 
\usepackage{graphicx,psfrag} 
\usepackage{fancyhdr}
\usepackage{color} 
\usepackage{geometry}
\usepackage{multirow}
\usepackage{listings}
\usepackage{enumerate}
\usepackage{leftidx} 
\usepackage{mathrsfs} 

\usepackage{listings}
\usepackage{color}

\definecolor{mygreen}{rgb}{0,0.6,0}
\definecolor{mygray}{rgb}{0.5,0.5,0.5}
\definecolor{mymauve}{rgb}{0.58,0,0.82}

\lstset{ %
  backgroundcolor=\color{white},   % choose the background color; you must add \usepackage{color} or \usepackage{xcolor}
  basicstyle=\footnotesize,        % the size of the fonts that are used for the code
  breakatwhitespace=false,         % sets if automatic breaks should only happen at whitespace
  breaklines=true,                 % sets automatic line breaking
  captionpos=b,                    % sets the caption-position to bottom
  commentstyle=\color{mygreen},    % comment style
  deletekeywords={...},            % if you want to delete keywords from the given language
  escapeinside={\%*}{*)},          % if you want to add LaTeX within your code
  extendedchars=true,              % lets you use non-ASCII characters; for 8-bits encodings only, does not work with UTF-8
  keepspaces=true,                 % keeps spaces in text, useful for keeping indentation of code (possibly needs columns=flexible)
  keywordstyle=\bfseries,       % keyword style
  language=SQL,                 % the language of the code
  morekeywords={*,...},            % if you want to add more keywords to the set
  numbers=none,                    % where to put the line-numbers; possible values are (none, left, right)
  numbersep=5pt,                   % how far the line-numbers are from the code
  numberstyle=\tiny\color{mygray}, % the style that is used for the line-numbers
  rulecolor=\color{black},         % if not set, the frame-color may be changed on line-breaks within not-black text (e.g. comments (green here))
  showspaces=false,                % show spaces everywhere adding particular underscores; it overrides 'showstringspaces'
  showstringspaces=false,          % underline spaces within strings only
  showtabs=false,                  % show tabs within strings adding particular underscores
  stepnumber=2,                    % the step between two line-numbers. If it's 1, each line will be numbered
  stringstyle=\color{mymauve},     % string literal style
  tabsize=2,                       % sets default tabsize to 2 spaces
  title=\lstname                   % show the filename of files included with \lstinputlisting; also try caption instead of title
}

\geometry{left=3.17cm,right=3.17cm,top=2.54cm,bottom=2.54cm}

\begin{document}

\pagestyle{fancy}
\rfoot{\thepage}
\rhead{\bfseries Database System Concept}
\setlength{\parskip}{0.7ex plus0.2ex minus0.2ex}
\cfoot{\empty}
\lhead{\empty}


\title{Assignment 8}
\author{Qinglin Li, 5110309074}
\date{}
\maketitle

\headheight 3pt
\thispagestyle{fancy}
\section*{Problem 1}
Use the B+-tree to find the first tuple with $branch\_name = \text{``Downtown"}$ and $branch\_city < \text{``Brooklyn"}$, then scan the following tuples and check whether each tuple satisfies the condition on $assets$. Stop the scanning when $branch\_name \neq \text{``Downtown"}$ and $branch\_city \geq \text{``Brooklyn"}$

\section*{Problem 2}
In a \textbf{serial schedule}, no transaction starts until a running transaction has ended.\\
A \textbf{serializable schedule} is a schedule that is equivalent (in its outcome) to a serial schedule

\section*{Problem 3}
\begin{enumerate}[a.]
\item
If we execute T1 first, the result should be $A=0$ and $B=1$\\
If we execute T2 first, the result should be $A=1$ and $B=0$\\
So in both case, the consistency requirement is satisfied.
\item ~\\
\begin{tabular}{|c|c|}
\hline 
T1 & T2 \\ 
\hline 
Read(A) & • \\ 
\hline 
• & Read(B) \\ 
\hline 
Read(B) & • \\ 
\hline 
• & Read(A) \\ 
\hline 
If A=0 then B:=B+1 & • \\ 
\hline 
• & If B=0 then A:=A+1 \\ 
\hline 
Write(B) & • \\ 
\hline 
• & Write(A) \\ 
\hline 
\end{tabular} 
\item 
No.\\
Because If we want to produce a serializable result, we must execute the first read instruction after the other's write.

\end{enumerate}
\section*{Problem 4}
\begin{enumerate}
\item
To improve throughput and resource utilization.
\item
To Reduce waiting time.
\end{enumerate}
\section*{Problem 5}
\begin{enumerate}[a.]
\item
T31:
\begin{itemize}%[label={}]
\item[] lock-S(A)
\item[] read(A)
\item[] lock-X(B)
\item[] read(B)
\item[] if A = 0 then B:= B + 1 
\item[] write(B)
\item[] unlock(A)
\item[] unlock(B)
\end{itemize}
T32:
\begin{itemize}
\item[] lock-S(B)
\item[]read(B)
\item[]lock-X(A)
\item[]read(A)
\item[]if B = 0 then A := A + 1
\item[]write(A)
\item[]unlock(B)
\item[]unlock(A)
\end{itemize}
\item
Yes.\\
T31 acquire the lock on A first and then T32 acquire the lock on B.
\end{enumerate}

\section*{Problem 6}
Advantage: It produces only cascadeless schedules, recovery is very easy.\\
Disadvantage: The set of schedules obtainable is a subset of those obtainable from plain two phase locking, thus concurrency is reduced\\

\section*{Problem 7}
Suppose a transaction is rolled back because of a newer transaction's reading or
writing the data which it plans to write. If the rolled back transaction
is re-introduced with the same timestamp, the transaction should be rolled back again because of the same reason and the process would never end. 

\section*{Problem 8}
A transaction may become the victim of deadlock-prevention rollback arbitrarily many times, thus creating a potential starvation situation.

\section*{Problem 9}
Volatile storage is storage which fails after a power failure. Caches, main memories are volatile storage.\\
Non-volatile storage is storage which retains its content after power failures. Magnetic disk, tapes are non-volatile storage.\\
Stable storage is storage which survives any kind of failure. This type of storage can only be approximated by replicating data.\\~\\
Volatile memory is the fastest and non-volatile storage is slower. Stable storage is the slowest because of the data replication.

\section*{Problem 10}
\begin{enumerate}[a.]
\item 
Stable storage cannot really be implemented because all storage devices are made of hardware, and all hardware is vulnerable to mechanical or electronic device failures.
\item 
 Database systems approximate stable storage by writing data to multiple storage devices simultaneously. Even if one of the devices crashes, the data will still be available on a different device.
\end{enumerate}

\section*{Problem 11}
In a system crash, the CPU goes down, and disk may also crash. But stable-storage at the site is assumed to survive system crashes. \\
In a disaster, everything at a site is destroyed. 
\end{document}

