\documentclass[a4paper,11pt]{article}
\usepackage{amsmath,amsthm,amsfonts,amssymb,bm} 
\usepackage{graphicx,psfrag} 
\usepackage{fancyhdr}
\usepackage{color} 
\usepackage{geometry}
\usepackage{multirow}
\usepackage{listings}
\usepackage{enumerate}
\usepackage{leftidx} 
\usepackage{mathrsfs} 
\usepackage{xeCJK}

\usepackage{listings}
\usepackage{color}

\definecolor{mygreen}{rgb}{0,0.6,0}
\definecolor{mygray}{rgb}{0.5,0.5,0.5}
\definecolor{mymauve}{rgb}{0.58,0,0.82}

\lstset{ %
  backgroundcolor=\color{white},   % choose the background color; you must add \usepackage{color} or \usepackage{xcolor}
  basicstyle=\footnotesize,        % the size of the fonts that are used for the code
  breakatwhitespace=false,         % sets if automatic breaks should only happen at whitespace
  breaklines=true,                 % sets automatic line breaking
  captionpos=b,                    % sets the caption-position to bottom
  commentstyle=\color{mygreen},    % comment style
  deletekeywords={...},            % if you want to delete keywords from the given language
  escapeinside={\%*}{*)},          % if you want to add LaTeX within your code
  extendedchars=true,              % lets you use non-ASCII characters; for 8-bits encodings only, does not work with UTF-8
  keepspaces=true,                 % keeps spaces in text, useful for keeping indentation of code (possibly needs columns=flexible)
  keywordstyle=\bfseries,       % keyword style
  language=SQL,                 % the language of the code
  morekeywords={*,...},            % if you want to add more keywords to the set
  numbers=none,                    % where to put the line-numbers; possible values are (none, left, right)
  numbersep=5pt,                   % how far the line-numbers are from the code
  numberstyle=\tiny\color{mygray}, % the style that is used for the line-numbers
  rulecolor=\color{black},         % if not set, the frame-color may be changed on line-breaks within not-black text (e.g. comments (green here))
  showspaces=false,                % show spaces everywhere adding particular underscores; it overrides 'showstringspaces'
  showstringspaces=false,          % underline spaces within strings only
  showtabs=false,                  % show tabs within strings adding particular underscores
  stepnumber=2,                    % the step between two line-numbers. If it's 1, each line will be numbered
  stringstyle=\color{mymauve},     % string literal style
  tabsize=2,                       % sets default tabsize to 2 spaces
  title=\lstname                   % show the filename of files included with \lstinputlisting; also try caption instead of title
}
\setCJKmainfont{Kai}
\geometry{left=3.17cm,right=3.17cm,top=2.54cm,bottom=2.54cm}

\begin{document}

\pagestyle{fancy}
\rfoot{\thepage}
\rhead{\bfseries Database System Concept}
\setlength{\parskip}{0.7ex plus0.2ex minus0.2ex}
\cfoot{\empty}
\lhead{\empty}


\title{Assignment 3}
\author{Qinglin Li, 5110309074}
\date{}
\maketitle

%\headheight 3pt
\thispagestyle{fancy}

\section*{Problem 1}
\begin{itemize}
\item
\textbf{Atomicity}: Each transaction is "all or nothing"---if one part of the transaction fails, the entire transaction fails, and the database state is left unchanged.\\
Example: Assume that a transaction attempts to subtract 10 from A and add 10 to B. If after removing 10 from A, the system crashes. Then when the system recovers, either A remains the old value or 10 is added to B.
\item
\textbf{Consistency}: Any transaction will bring the database from one valid state to another. Any data written to the database must be valid according to all defined rules.\\
Example: Assume in a table there is a contraint requires the sum of A and B to be 100. If a transaction attempts to subtract 10 from A without modifying B, the transaction should not be commited.
\item
\textbf{Isolation}: The concurrent execution of transactions results in a system state that would be obtained if transactions were executed serially.\\
Example: Assume two transactions T1 and T2 both attempts to subtract 10 from A and add 10 to B. The order of two transactions should either be 
\begin{enumerate}
\item T1 subtracts 10 from A.
\item T1 adds 10 to B.
\item T2 subtracts 10 from B.
\item T2 adds 10 to A.
\end{enumerate}
or 
\begin{enumerate}
\item T2 subtracts 10 from A.
\item T2 adds 10 to B.
\item T1 subtracts 10 from B.
\item T1 adds 10 to A.
\end{enumerate}
\item
\textbf{Durability}: Transactions that have committed will survive permanently.\\
Example: A transaction is commited, but when the changes are still queued in the disk buffer waiting to be commited to the disk, the power fails. Then after the system recovers, the uncommited changes should be commited to the disk.
\end{itemize}
\newpage
\section*{Problem 2}
\begin{lstlisting}
create table loan(
    loan_number char(10),
    branch_name char(15),
    amount integer,
    primary key (loan_number),
    foreign key (branch_name) references branch
)

create table borrower(
    customer_name char(20),
    loan_number char(10),
    primary key (customer_name, loan_number),
    foreign key (customer_name) references customer,
    foreign key (loan_number) references loan
)
\end{lstlisting}

\section*{Problem 3}
\begin{enumerate}[a.]
\item \textbf{foreign key} (name) \textbf{references} salaried\_workder or hourly\_worker
\item use general assertions
\begin{lstlisting}
create assertion name_ref 
check(
    not exists (
        select * from address
        where name not in (
            select name from salaried_workder 
            union
            select name from hourly_worker
        )    
    )
)
\end{lstlisting}
\end{enumerate}
\section*{Problem 4}
\begin{enumerate}[a.]
\item
\begin{lstlisting}
select company from works
group by company
having avg(salary) > (
    select avg(salary) from works 
    where company = 'First Bank Corporation'
)
\end{lstlisting}
\newpage
\item It should work with MySQL
\begin{lstlisting}
delimiter //
create function avg_salary (company_name varchar(20))
returns double
begin
    declare ret double;
    select avg(salary) into ret from works where company = company_name;
    return ret;
end;//
delimiter ;
select name from works 
where avg_salary(name) > avg_salary('First Bank Corporation');
\end{lstlisting}
\end{enumerate}
\section*{Problem 5}
\begin{enumerate}[a.]
\item $\{t|\exists p\in r (t[A]=p[A])\}$
\item $\{t|t\in r\wedge t[B]=17\}$
\item $\{t|\exists p\in r, \exists q\in s (t[A] = p[A]\wedge t[B] = p[B]\wedge t[C] = p[C]\wedge t[D] = q[D]\wedge t[E] = q[E]\wedge t[F] = q[F])\}$
\item $\{t|\exists p\in r, \exists q\in s (t[A] = p[A]\wedge t[F] = q[F]\wedge p[C] = q[D]\}$
\end{enumerate}

\section*{Problem 6}
\begin{enumerate}[a.]
\item $\Pi_A(\sigma_{B=17}(r))$
\item $r\bowtie s$
\item $\Pi_A(r)\bigcup(s\div \Pi_C(s))$
\item $\Pi_A\big(\sigma_{B>D}(r \bowtie s \bowtie \rho_{t(C,D)}(r))\big)$
\end{enumerate}

\section*{Problem 7}
\begin{enumerate}[a.]
\item
$\{t|\exists r\in R \exists s\in S(r[A]=s[A]\wedge t[A]=r[A]\wedge t[B]=r[B]\wedge t[C]=s[C])\vee \exists s\in S(\neg \exists r\in R(r[A]=s[A])\wedge t[A]=s[A]\wedge t[C]=s[C]\wedge t[B]=null)\}$
\item
$\{t|\exists r\in  R\exists s\in  S(r[A]=s[A]\wedge t[A]=r[A]\wedge t[B]=r[B]\wedge t[C]=s[C]) \vee  \exists r \in   R(\neg \exists s \in  S(r[A]=s[A])\wedge t[A]=r[A]\wedge t[B]=r[B]\wedge t[C]=null) \vee  \exists s \in   S(\neg \exists r \in  R(r[A]=s[A]) \wedge  t[A]=s[A] \wedge  t[C]=s[C] \wedge  t[B]=null)\}$
\item
$\{t|\exists r\in  R\exists s\in  S(r[A]=s[A]\wedge t[A]=r[A]\wedge t[B]=r[B]\wedge t[C]=s[C]) \vee  \exists r \in   R(\neg \exists s \in  S(r[A]=s[A]) \wedge  t[A]=r[A] \wedge  t[B]=r[B] \wedge  t[C]=null)\}$
\end{enumerate}
\end{document}
